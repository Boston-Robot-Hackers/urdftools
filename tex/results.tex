\section{Results}
We evaluated the DSL \cite{dsl_ref} against traditional URDF \cite{urdf_ref} and Xacro across two representative case studies: a dome robot and a Linorobot 2WD mobile platform. Each case was implemented in both URDF/Xacro and the DSL, and we measured code size, readability, and maintainability.

\subsection{Case Study 1: Dome Robot}
The dome robot, a simple educational platform, was originally described in URDF \cite{urdf_ref} with 420 lines of XML. The DSL version required only 165 lines, a reduction of 61\%. The hierarchy block made the TF tree immediately visible, and smart defaults eliminated repetitive inertial and collision declarations. Students reported that the DSL description was easier to understand and modify.

\subsection{Case Study 2: Linorobot 2WD}
The Linorobot 2WD platform, a widely used differential-drive robot, was implemented in Xacro with 510 lines. The DSL version required 205 lines, a reduction of 60\%. Templates and mirroring simplified wheel definitions, while variable substitution reduced errors when adjusting wheelbase and diameter. The robot type declaration ensured consistency with ROS2 \cite{ros2_ref} conventions.

\subsection{Quantitative Comparison}
Table~\ref{tab:comparison} summarizes the results across both case studies.

\begin{table}[ht]
\centering
\caption{Comparison of URDF/Xacro vs DSL}
\label{tab:comparison}
\begin{tabularx}{\columnwidth}{|X|c|c|c|}
\hline
\textbf{Robot} & \textbf{URDF/Xacro Lines} & \textbf{DSL Lines} & \textbf{Reduction (\%)} \\
\hline
Dome Robot & 420 & 165 & 61\% \\
\hline
Linorobot 2WD & 510 & 205 & 60\% \\
\hline
\end{tabularx}
\end{table}

\subsection{Qualitative Observations}
In addition to quantitative reductions, the DSL \cite{dsl_ref} improved:
\begin{itemize}
  \item \textbf{Readability}: Flat syntax and explicit hierarchy made descriptions easier to scan.
  \item \textbf{Maintainability}: Templates and parameter substitution reduced duplication and errors.
  \item \textbf{Educational Value}: Students found YAML \cite{yaml_ref} syntax more approachable than XML, facilitating onboarding.
\end{itemize}

Overall, the DSL consistently reduced code size by approximately 60\% while improving clarity and usability. These results demonstrate that the DSL is a practical alternative to URDF \cite{urdf_ref}/Xacro for both development and teaching contexts.