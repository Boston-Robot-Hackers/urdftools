\section{Implementation}
The YAML-based DSL \cite{yaml_ref,dsl_ref} was implemented as a lightweight compiler that translates concise YAML descriptions into canonical URDF \cite{urdf_ref} files. The implementation emphasizes reproducibility, compatibility with ROS2 \cite{ros2_ref} tooling, and ease of integration into existing workflows.

\subsection{Compiler Pipeline}
The compiler follows a three-stage pipeline:
\begin{enumerate}
  \item \textbf{Parsing}: YAML input is parsed into an abstract syntax tree (AST). Parameters, templates, and hierarchy declarations are resolved at this stage.
  \item \textbf{Expansion}: Templates and generators are expanded, variable substitutions are applied, and smart defaults are injected.
  \item \textbf{Emission}: The AST is serialized into URDF XML. Each DSL construct maps directly to a URDF element, ensuring compatibility with ROS2 visualization, simulation, and control tools.
\end{enumerate}

\subsection{Validation}
Validation is performed both during parsing and emission:
\begin{itemize}
  \item \textbf{Schema checks}: Each block (links, joints, hierarchy, type) is validated against a schema to ensure required fields are present.
  \item \textbf{Consistency checks}: Hierarchy declarations are verified against defined links and joints. Robot type declarations are checked against ROS conventions (e.g., REP-105, REP-120).
  \item \textbf{Numerical checks}: Mass, inertia, and geometry parameters are validated for physical plausibility.
\end{itemize}

\subsection{Syntax Rules}
The DSL \cite{dsl_ref} introduces a small set of syntax rules designed for clarity:
\begin{itemize}
  \item \textbf{Flat fields}: Each link or joint is described with top-level fields rather than nested XML.
  \item \textbf{Explicit hierarchy}: Parent-child relationships are declared in a dedicated hierarchy block.
  \item \textbf{Parameter substitution}: Parameters can be defined once and reused with arithmetic expressions.
  \item \textbf{Templates}: Components can be defined once and instantiated multiple times, with support for mirroring and instancing.
  \item \textbf{Materials}: Semantic names replace verbose RGBA definitions, though custom materials remain supported.
\end{itemize}

\subsection{Integration with ROS2}
The compiler outputs standard URDF \cite{urdf_ref} files that can be consumed by ROS2 \cite{ros2_ref} packages such as \texttt{robot\_state\_publisher}, \texttt{rviz2}, and Gazebo. Because the DSL \cite{dsl_ref} is a strict superset of URDF semantics, no modifications to downstream tools are required. Developers can continue to use existing ROS2 workflows while benefiting from the conciseness and readability of the DSL.

\subsection{Implementation Footprint}
The compiler is implemented in fewer than 1,000 lines of Python, relying on widely available libraries for YAML \cite{yaml_ref} parsing and XML generation. This makes the tool lightweight, easy to audit, and simple to extend. The design emphasizes transparency and reproducibility, aligning with the educational goals of the project.