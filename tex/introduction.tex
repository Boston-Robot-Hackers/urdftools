\section{Introduction}
Robot description formats are foundational to the ROS2 ecosystem \cite{ros2_ref}, enabling simulation, visualization, and control. The Unified Robot Description Format (URDF) \cite{urdf_ref}, an XML-based specification, has been the de facto standard for over a decade. URDF provides a detailed representation of links, joints, inertial properties, and visual geometry, but its verbosity imposes significant costs. Developers must manage repetitive structures, nested hierarchies, and manual inertia calculations, while educators face challenges in teaching newcomers to navigate XML syntax.

Xacro, an extension of URDF \cite{urdf_ref} that introduces macros and parameterization, alleviates some repetition but remains tied to XML’s nested structure. Writing and maintaining Xacro files is often described as “hairy,” with readability and maintainability compromised by complex macro expansions. As robots grow in complexity, these limitations hinder both rapid prototyping and reproducible teaching workflows.

In this paper, we introduce a YAML-based Domain-Specific Language (DSL) \cite{yaml_ref,dsl_ref} that compiles directly to URDF \cite{urdf_ref}. Our DSL is designed around seven principles: flat syntax, explicit hierarchy sections, smart defaults, templates and generators, variable substitution, named material libraries, and robot type declarations. Together, these principles reduce boilerplate, expose the robot’s TF tree at a glance, and enable concise yet expressive descriptions. We demonstrate the DSL through two case studies—a dome robot and a Linorobot 2WD platform—showing consistent 60\% reductions in code size and improved clarity. We argue that this approach represents a practical step toward more maintainable and teachable robot description formats in ROS2 \cite{ros2_ref}.