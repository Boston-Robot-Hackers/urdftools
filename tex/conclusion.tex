\section{Conclusion and Future Work}
We presented a YAML-based Domain-Specific Language (DSL) \cite{yaml_ref,dsl_ref} for concise robot description in ROS2 \cite{ros2_ref}. By compiling directly to URDF \cite{urdf_ref}, the DSL reduces verbosity, improves readability, and enables self-contained robot descriptions. Case studies demonstrated consistent 60\% reductions in code size and improved clarity.

The DSL contributes both technically and educationally:
\begin{itemize}
  \item \textbf{Technical impact}: It provides a maintainable, modular alternative to URDF/Xacro while remaining fully compatible with ROS2 tooling.
  \item \textbf{Educational impact}: Its YAML syntax lowers the barrier to entry for students and newcomers, making robot description more approachable.
\end{itemize}

Future work includes:
\begin{itemize}
  \item Extending the DSL with semantic validation against additional REP standards.
  \item Exploring integration with simulation environments beyond Gazebo.
  \item Conducting user studies to measure onboarding time and error rates compared to URDF \cite{urdf_ref}/Xacro.
  \item Investigating community-driven extensions to support new robot types and advanced generators.
\end{itemize}

In summary, the DSL demonstrates that concise, human-readable formats can coexist with established XML-based specifications. By bridging educational accessibility with technical rigor, it offers a pathway toward more maintainable and inclusive robot description practices in ROS2 \cite{ros2_ref}.